% interactcadsample.tex
% v1.03 - April 2017

\documentclass[]{interact}

\usepackage{epstopdf}% To incorporate .eps illustrations using PDFLaTeX, etc.
\usepackage{subfigure}% Support for small, `sub' figures and tables
%\usepackage[nolists,tablesfirst]{endfloat}% To `separate' figures and tables from text if required

\usepackage{natbib}% Citation support using natbib.sty
\bibpunct[, ]{(}{)}{;}{a}{}{,}% Citation support using natbib.sty
\renewcommand\bibfont{\fontsize{10}{12}\selectfont}% Bibliography support using natbib.sty

\theoremstyle{plain}% Theorem-like structures provided by amsthm.sty
\newtheorem{theorem}{Theorem}[section]
\newtheorem{lemma}[theorem]{Lemma}
\newtheorem{corollary}[theorem]{Corollary}
\newtheorem{proposition}[theorem]{Proposition}

\theoremstyle{definition}
\newtheorem{definition}[theorem]{Definition}
\newtheorem{example}[theorem]{Example}

\theoremstyle{remark}
\newtheorem{remark}{Remark}
\newtheorem{notation}{Notation}

% see https://stackoverflow.com/a/47122900


\usepackage{hyperref}
\usepackage[utf8]{inputenc}
\def\tightlist{}
\usepackage[usenames,dvipsnames]{color}
\newcommand{\er}[1]{\textcolor{Orange}{#1}}
\newcommand{\svp}[1]{\textcolor{Green}{#1}}
\newcommand{\rh}[1]{\textcolor{Plum}{#1}}

\begin{document}

\articletype{JSM 2021 Student Paper Award (ASA sections on Statistical Computing and
Statistical Graphics)}

\title{Perception of exponentially increasing data displayed on a log scale}


\author{\name{Emily A. Robinson$^{a}$, Reka Howard$^{a}$, Susan VanderPlas$^{a}$}
\affil{$^{a}$Department of Statistics, University of Nebraska - Lincoln,}
}

\thanks{CONTACT Emily A. Robinson. Email: \href{mailto:emily.robinson@huskers.unl.edu}{\nolinkurl{emily.robinson@huskers.unl.edu}}, Reka Howard. Email: \href{mailto:rekahoward@unl.edu}{\nolinkurl{rekahoward@unl.edu}}, Susan VanderPlas. Email: \href{mailto:susan.vanderplas@unl.edu}{\nolinkurl{susan.vanderplas@unl.edu}}}

\maketitle

\begin{abstract}
Log scales are often used to display data over several orders of
magnitude within one graph. During the COVID pandemic, we've seen both
the benefits and the pitfalls of using log scales to display data. This
paper aims to\ldots{}
\end{abstract}

\begin{keywords}
Exponential; Log; Visual Inference; Perception
\end{keywords}

\hypertarget{introduction-and-background}{%
\section{Introduction and
Background}\label{introduction-and-background}}

\begin{itemize}
\tightlist
\item
  Why Graphics? (communication to the public, technological advances,
  need for research on graphics)
\end{itemize}

\textcolor{Orange}{
Graphics are a useful tool for displaying and communicating information. 
Researchers include graphics to communicate their results in scientific publications and news sources rely on graphics to convey news stories to the public. 
During the onset of the novel coronavirus - covid19 - pandemic, we saw an influx of dashboards being developed to display case counts, transmission rates, and outbreak regions \citep{lisa_charlotte_2020}.
As a result, people began subscribing to news sources involved in graphically tracking the coronavirus (example John Burn-Murdoch Financial Times - CITE THIS) and gaining more exposure to the use of graphics. 
Many of these graphics helped guide decision makers to implement policies such as shut-downs or mandated mask wearing. 
Better software has meant easier and more flexible drawing, consistent themes, and higher standards.
Consequentially, we must develop a set of principles to help us actively choose which of many possible graphics to draw \citep{unwin_why_2020}.
}

\begin{itemize}
\tightlist
\item
  Introduce Log Scales (what are they used for, where are they used
  (ecological data, covid, etc.))
\end{itemize}

\textcolor{Orange}{
When faced with data which spans several orders of magnitude, we must decide whether to show the data on its original scale (compressing the smaller magnitudes into relatively little area) or to transform the scale and alter the contextual appearance of the data. [EXAMPLE HERE]
One common graphical display choice is the use of log scales used to display data over several orders of magnitude within one graph. 
Logarithms convert multiplicative relationships to additive ones, providing an elegant way to span many orders of magnitude, to show elasticities and other proportional changes, and to linearize power laws \citep{menge_logarithmic_2018}. 
When presenting log-scaled data, it is possible to use either untransformed values (for example, values of 1, 10 and 100 are equally spaced along the axis) or log-transformed values (for example, 0, 1, and 2). 
We have recently experienced the benefits and pitfalls of using log-scales as covid-19 dashboards displayed case count data on both the log and linear scale \citep{wade_fagen_ulmschneider_2020}. 
INSERT BENEFITS AND PITFALLS OF LOG SCALES HERE. 
While COVID-19 is the most well known example, log-scales have been used to display data in ecological research, etc. 
PUT OTHER AREAS HERE.
}

\begin{itemize}
\tightlist
\item
  Previous exponential (log/linear scale) studies (literature review).
\end{itemize}

\textcolor{Orange}{
Previous research suggests our perception and mapping of numbers to a numberline proceeds logrithmically at first and transitions to linear later in development. 
The logarithmic mapping is more noticible in larger numbers as the transition to linear mapping occurs first for small numbers in young children and later for larger numbers \citep{varshney_why_2013, siegler_numerical_2017, dehaeneLogLinearDistinct2008}.
}

\begin{itemize}
\tightlist
\item
  Visual Inference (what is it? how do we use it? etc.)
\end{itemize}

\textcolor{Orange}{Statistical lineups have previously been utilized in graphical experiments to quantify... \citep{buja_statistical_2009, wickham2010graphical, hofmann_graphical_2012, majumder_validation_2013, vanderplas_clusters_2017}.
\cite{vanderplas_statistical_nodate} provides an approach for calculating visual p-values
}

\begin{itemize}
\tightlist
\item
  What is new in this paper.
\end{itemize}

\hypertarget{data-generation}{%
\section{Data Generation}\label{data-generation}}

\textcolor{Orange}{
The most common type of lineup used in graphical experiments is a standard lineup containing one "target" dataset embeded within a set of null datasets. 
One way to generate the null datasets when working with real data is through the use of permutation. In this study, the target dataset was generated by model A while the null datasets were generated by model B.
}

\hypertarget{exponential-model}{%
\subsection{Exponential Model}\label{exponential-model}}

\textcolor{Orange}{}

\hypertarget{parameter-selection}{%
\subsection{Parameter Selection}\label{parameter-selection}}

\begin{itemize}
\tightlist
\item
  Use of lack of fit statistic.
\item
  Mapping parameter selections to what we see visually.
\item
  Curvature (Easy/Medium/Hard)
\end{itemize}

\hypertarget{study-design}{%
\section{Study Design}\label{study-design}}

\hypertarget{lineup-setup}{%
\subsection{Lineup Setup}\label{lineup-setup}}

\hypertarget{participant-recruitment}{%
\subsection{Participant Recruitment}\label{participant-recruitment}}

\textcolor{Orange}{Participants were recruited from Reddit. GIVE SUMMARY DESCRIPTIVE STATISTICS OF PARTICIPANT DEMOGRAPHICS.}

\hypertarget{task-description}{%
\subsection{Task Description}\label{task-description}}

\begin{itemize}
\item
  Lineup Task

  \begin{itemize}
  \item
    \textcolor{Orange}{The goal of this is to test an individuals ability to perceptually differentiate exponentially increasing data with differing rates of change on both the linear and log scale.}
  \end{itemize}
\end{itemize}

\hypertarget{results}{%
\section{Results}\label{results}}

\hypertarget{effect-of-curvature}{%
\subsection{Effect of Curvature}\label{effect-of-curvature}}

\hypertarget{effect-of-variability}{%
\subsection{Effect of Variability}\label{effect-of-variability}}

\hypertarget{linear-vs-log}{%
\subsection{Linear vs Log}\label{linear-vs-log}}

\hypertarget{participant-reasoning}{%
\subsection{Participant Reasoning}\label{participant-reasoning}}

\hypertarget{discussion}{%
\section{Discussion}\label{discussion}}

\hypertarget{conclusion}{%
\subsection{Conclusion}\label{conclusion}}

\hypertarget{future-research}{%
\subsection{Future Research}\label{future-research}}

\begin{itemize}
\item
  What we learned from lineups but what we still want to learn.
\item
  You draw it

  \begin{itemize}
  \item
    \textcolor{Orange}{\citep{mosteller_eye_1981} designed and carried out an empirical investigation to explore properties of lines fitted by eye. The researchers found that students tended to fit the slope of the first principal component or major axis (the line that minimizes the sum of squares of perpendicular rather than vertical distances) and that students who gave steep slopes for one data set also tended to give steep slopes on the others. Interestingly, the individual-to-individual variability in slope and in intercept was near the standard error provided by least squares for the four data sets.}
  \item
    \textcolor{Orange}{The goal of this task is to test an individuals ability to make predictions for exponentially increasing data.}
  \item
    \textcolor{Orange}{Previous literature suggests that we tend to underestimate predictions of exponentially increasing data.\citep{jones_generalized_1979, jones_polynomial_1977, wagenaar_extrapolation_1978}}
  \item
    \textcolor{Orange}{The idea for this task was inspired by the New York Times "You Draw It" page which is fun to check out.}
  \end{itemize}
\item
  Estimation

  \begin{itemize}
  \item
    \textcolor{Orange}{This tests an individuals ability to translate a graph of exponentially increasing data into real value quantities. We then ask individuals to extend their estimates by making comparisons across levels of the independent variable.}
  \item
    \textcolor{Orange}{\citep{friel_making_2001} emphasize the importance of graph comprehension proposing that the graph construction plays a role in the ability to read and interpret graphs.}
  \end{itemize}
\end{itemize}

\hypertarget{supplementary-materials}{%
\section*{Supplementary Materials}\label{supplementary-materials}}
\addcontentsline{toc}{section}{Supplementary Materials}

\hypertarget{acknowledgements}{%
\section*{Acknowledgement(s)}\label{acknowledgements}}
\addcontentsline{toc}{section}{Acknowledgement(s)}

\bibliographystyle{tfcad}
\bibliography{references.bib}




\end{document}
