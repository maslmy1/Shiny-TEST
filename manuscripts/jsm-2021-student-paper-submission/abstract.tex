  Log scales are often used to display data over several orders of magnitude within one graph. During the COVID-19 pandemic, we have seen both the benefits and the pitfalls of using log scales to display case counts. In this paper, we explore the use of linear and log scales to determine whether our ability to notice differences in exponentially increasing trends is impacted by the scales choice. We conducted a visual inference experiment in which participants were shown a series of lineup plots (consisting of 19 null panels and 1 target panel generated by differing model parameters) and asked to identify the panel that was most different from the others. We found that displaying exponentially increasing data on a log scale improved the accuracy of differentiating between models with slight curvature differences, particularly when identifying an exponential curve with less curvature than others. When there was a larger curvature difference, participants accurately identified the target panel on both the linear and log scale. An exception occurs when identifying a plot with more curvature than the surrounding plots, indicating that it is is more difficult to say something has less curvature, but easy to say that something has more curvature.